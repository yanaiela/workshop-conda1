The open exchange of ideas, the freedom of thought and expression, and respectful scientific debate are central to the aims and goals of a ACL conference. These require a community and an environment that recognizes the inherent worth of every person and group, that fosters dignity, understanding, and mutual respect, and that embraces diversity. For these reasons, ACL is dedicated to providing a harassment-free experience for participants at our events and in our programs.
Harassment and hostile behavior are unwelcome at any ACL conference. This includes: speech or be- havior (including in public presentations and on-line discourse) that intimidates, creates discomfort, or interferes with a person’s participation or opportunity for participation in the conference. We aim for ACL conferences to be an environment where harassment in any form does not happen, including but not lim- ited to: harassment based on race, gender, religion, age, color, national origin, ancestry, disability, sexual orientation, or gender identity. Harassment includes degrading verbal comments, deliberate intimidation, stalking, harassing photography or recording, inappropriate physical contact, and unwelcome sexual at- tention.
It is the responsibility of the community as a whole to promote an inclusive and positive environment for our scholarly activities. In addition, any participant who experiences harassment or hostile behavior may contact any current member of the ACL Board or contact Priscilla Rasmussen, who is usually available at the registration desk of the conference. Please be assured that if you approach us, your concerns will be kept in strict confidence, and we will consult with you on any actions taken.
The ACL board members are listed at:
https://www.aclweb.org/portal/about
The full policy and its implementation is defined at:
https://www.aclweb.org/adminwiki/index.php?title=Anti\-Harassment\_Policy